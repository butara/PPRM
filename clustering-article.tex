\documentclass[]{IEEEphot}
\usepackage[utf8]{inputenc}
\usepackage[parfill]{parskip}

\jvol{xx}
\jnum{xx}
\jmonth{November}
\pubyear{2011}

\begin{document}
\title{Machine Learning: Clustering algorithms review}
\author{Mentor, Primož, Matej}
\affil{Fakulteta za računalništvo in informatiko}  
\maketitle

\begin{abstract}
Abstract is missing!
\end{abstract}

\begin{IEEEkeywords}
Machine learning, clustering, k-means, ECMC, Expectation Maximization.
\end{IEEEkeywords}

\section{Introduction}
What is clustering? It's an unsupervised machine learning task. Computationally difficult
(NP-hard). Heuristic algorithms. Iterative refinements.

\section{Related work}

\newpage
\section{Algorithms}
\subsection{k-means}
k-means clustering (KMC) is a simple and well known algorithm. It is usually very
fast (commonly run multiple times with different starting conditions), but tends
to produce spherical clusters with similar size, which can be undesirable.
k-means is often used as a preprocessing step for other algorithms. The standard
algorithm was first proposed by Stuart Lloyd in 1957 and published later in 1982.

The number of clusters we want to find in data is fixed a priori ($k$).
The algorithm iteratively refines clusters' centroids,
until no more refinements are possible (centroids do not move anymore)
or max number of iterations has been reached. $m_1^{(1)}$, $m_2^{(1)}$, ..., $m_k^{(1)}$
denote centroids (also called means). The superscript index denotes the current iteration.
$S_i^{(t)}$ is a cluster (set of examples) with the corresponding centroid $m_i^{(t)}$.
There are two alternating steps in the algorithm:

\begin{itemize}
\item \textbf{Assignment step}\\
  Each example is assigned to the cluster with the closest centroid:\\
  $S_i^{(t)} = \{x_j : ||x_j - m_i^{(t)}|| \le ||x_j - m_l^{(t)}|| \forall l = 1, ..., k\}$
\\
\item \textbf{Update step}\\
  New means are calculated and set as centroids:\\
  $m_i^{(t+1)} = \frac{1}{|S_i^{(t)}|} \displaystyle\sum\limits_{x_j \in S_i^{(t)}}{x_j}$
\end{itemize}

There are two common ways to select the initial centroids: random seed and random partition.
The random seed method randomly chooses k observations from the data set
and uses these as the initial means. Random partition
assigns each example into one of the k clusters randomly.
The random seed method tends to spread the initial means out, while random partition
places all of them close to the center of the data set.

There are also better approaches to initialization: k-means++ algorithm specifies a
procedure to initialize the cluster centers before proceeding with the standard k-means
optimization iterations. With the k-means++ initialization, the algorithm is guaranteed
to find a solution that is $O(\log k)$ competitive to the optimal k-means solution.
Although this initial selection takes extra time, it speeds up the convergence and 
therefore actually lowers overall computation time.

Although it can be proved that the KMC algorithm will always terminate (converge),
it does not necessarily find the most optimal configuration,
corresponding to the global objective function minimum. The algorithm is also sensitive
to the initial selected cluster centroids but it can be run multiple times to reduce this effect.

Another interesting fact is that, in the worst case, k-means can be very slow to converge:
in particular it has been shown that there exist certain point sets,
even in 2 dimensions, on which k-means takes exponential time to converge.
These point sets do not seem to arise in practice: this is supported by the
fact that the smoothed time complexity of k-means is polynomial.

\subsection{Evolving Clustering Method}

\subsection{Expectation Maximization Gaussian Mixture Model}

\subsection{Cauchy-Schwarz Divergence Clustering}
KMC is an example of so-called \textit{parametric} methods of clustering. These
methods assume some knowledge about clusters' structure, which makes them less
suitable for many use cases. KMC performs badly when clusters are not hyperelliptical
because it implicitly assumes Gaussian cluster distributions. Any clustering method
that utilizes second order data statistics can produce only convex clusters (Jain et al., 2000).

Information theory has been successfully used in clustering by several researchers in recent years.
Various information-theoretic clustering metrics, such as entropy, mutual information and
Kullback-Leibler divergence were researched.

\section{Results}
\subsection{Method}
\subsection{Data preprocessing}

\section{Conclusions}

\section*{Acknowledgements}
Thanks to TODO.

\section*{References}

\end{document}